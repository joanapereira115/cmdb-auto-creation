%!TEX root = ../dissertation.tex


%import the necessary package with some options
\usepackage[acronym,nonumberlist,nomain]{glossaries}
%enable the following to avoid links from the acronym usage to the list
%\glsdisablehyper
%displays the first use of an acronym in italic
\defglsdisplayfirst[\acronymtype]{\emph{#1#4}}
%the style of the Glossary
\glossarystyle{listgroup}
% set the name for the acronym entries page
\renewcommand{\glossaryname}{Acronyms}
%this shall be the last thing in the acronym configuration!!
\makeglossaries
% here are the acronym entries

\newacronym{itil}{ITIL}{Information Technology Infrastructure Library}
\newacronym{cmdb}{CMDB}{Configuration Management Database}
\newacronym{ci}{CI}{Configuration Item}
\newacronym{iso}{ISO}{International Organization for Standardization}
\newacronym{ip}{IP}{Internet Protocol}
\newacronym{mac}{MAC}{Media Access Control}
\newacronym{icmp}{ICMP}{Internet Control Message Protocol}
\newacronym{snmp}{SNMP}{Simple Network Management Protocol}
\newacronym{lan}{LAN}{Local Area Network}
\newacronym{wan}{WAN}{Wide Area Netwok}
\newacronym{vpn}{VPN}{Virtual Private Network}
\newacronym{fdb}{FDB}{Forwarding Database}
\newacronym{arp}{ARP}{Address Resolution Protocol}
\newacronym{lldp}{LLDP}{Link Layer Discovery Protocol}
\newacronym{mib}{MIB}{Management Information Base}
\newacronym{tcp}{TCP}{Transmission Control Protocol}
\newacronym{udp}{UDP}{User Datagram Protocol}
\newacronym{stp}{STP}{Spanning Tree Protocol}
\newacronym{span}{SPAN}{Switched Port Analyzer}
\newacronym{rdp}{RDP}{Remote Desktop Protocol}
\newacronym{api}{API}{Application Programming Interface}
\newacronym{rest}{REST}{Representational State Transfer}
\newacronym{ti}{TI}{Tecnologias da Informação}
\newacronym{itsm}{ITSM}{IT Service Management}
\newacronym{osi}{OSI}{Open System Interconnection}
\newacronym{ssh}{SSH}{Secure Shell}
\newacronym{wmi}{WMI}{Windows Management Instrumentation}
\newacronym{winrm}{WinRM}{Windows Remote Management}
\newacronym{cdp}{CDP}{Cisco Discovery Protocol}
\newacronym{jmx}{JMX}{Java Management Extensions}
\newacronym{sql}{SQL}{Structured Query Language}
\newacronym{dns}{DNS}{Domain Name System}
\newacronym{nfs}{NFS}{Network File System}
\newacronym{ldap}{LDAP}{Lightweight Directory Access Protocol}
\newacronym{taddm}{TADDM}{Tivoli Application Dependency Discovery Manager}
\newacronym{ssd}{SSD}{Solid-State Drive}
\newacronym{das}{DAS}{Direct Attached Storage}
\newacronym{nas}{NAS}{Network Attached Storage}
\newacronym{san}{SAN}{Storage Area Network}
\newacronym{cpu}{CPU}{Central Processing Unit}
\newacronym{bios}{BIOS}{Basic Input/Output System}
\newacronym{json}{JSON}{JavaScript Object Notation}
\newacronym{http}{HTTP}{HyperText Transfer Protocol}
\newacronym{xml}{XML}{eXtensible Markup Language}
\newacronym{cim}{CIM}{Common Information Model}
\newacronym{svs}{SVS}{Service Value System}
\newacronym{dmtf}{DMTF}{Distributed Management Task Force}
\newacronym{uml}{UML}{Unified Modeling Language}

% these could go in an acronyms.tex file, and loaded with:
% \loadglsentries[\acronymtype]{Parts/Definitions/acronyms}
% when using this, you may want to remove 'nomain' from the package options
%% **MORE INFO** %%
%to add the acronyms list add the following where you want to print it:
%\printglossary[type=\acronymtype]
%\clearpage
%\thispagestyle{empty}
%to use an acronym:
%\gls{qps}
% compile the thesis in command line with the following command sequence:
% pdlatex dissertation.tex
% makeglossaries dissertation
% bibtex dissertation
% pdlatex dissertation.tex
% pdlatex dissertation.tex